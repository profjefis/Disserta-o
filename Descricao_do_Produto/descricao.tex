\section{A Sequ�ncia Did�tica}
Aqui nesta se��o comentarei sobre a montagem da sequ�ncia did�tica.

Total de aulas: 10

P�blico alvo: estudantes da terceira s�rie do ensino m�dio do CED 15 - Ceil�ndia.

Total de turmas: 4

\begin{itemize}
	\item 1� Aula: Pesquisa inicial - jogos digitais
	\item 2� Aula: Pr�-teste
	\item 3� Aula: 1� Aplica��o do jogo
	\item 4� Aula e 5� Aula: V�deo \aspas{O Discreto charme das part�culas elementares}
	\item 6� Aula e 7� Aula: Aula expositiva
	\item 8� Aula: Segunda aplica��o do jogo
	\item 9� Aula: P�s-teste
	\item 10� Aula: Pesquisa final - aplica��o do produto educacional
\end{itemize}

\section{O Jogo Digital}

Aqui ser� abordado os seguintes t�picos sobre o jogo: descri��o, estilo, din�mica, jogabilidade, plataforma de programa��o e arte.

%O jogo tem um estilo RPG/Aventura, no qual o jogador controla o personagem principal. 
%O jogo uma sua forma principal um quiz (perguntas e resposta) com um total de 20 perguntas %pertinentes � FPE. o foco das perguntas � a teoria, n�o foi dado �nfase � matem�tica
%Dinamica do jogo
