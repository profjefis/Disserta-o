\section{Metodologia}

Aqui ser� descrita a metodologia de aplica��o do produto educacional.

%Uma extens�o deste trabalho pode ser a implementa��o de um \emph{controle �timo global por %redund�ncia} da junta passiva do manipulador. Para isto pode-se fazer o planejamento %\emph{off-line} da trajet�ria das juntas de modo a minimizar a energia consumida. Alguns estudos %foram feitos nesse sentido, usando o Princ�pio M�nimo de Pontryagin, mas sem resultados %satisfat�rios at� o momento.

%\begin{quadro}
%	\centering
%	\caption{Teste2}
%	\begin{tabular}{|c|c|}\hline
%		Coluna 1 &amp; Coluna 2 \\ \hline
%		Conte�do 1 &amp; Conte�do 2 \\ \hline
%	\end{tabular}
%\end{quadro}

\section{Relato de Experi�ncia}

Esta parte constar� o relato de experi�ncia da aplica��o do produto. Ser� um relato aula a aula de toda a sequ�ncia de aplica��o, destacando os principais aspectos nos comportamento dos alunos durante a aplica��o.

\subsection{Aula 1}
\subsection{Aula 2}
\subsection{Aula 3}
\subsection{Aulas 4 e 5}
\subsection{Aulas 6 e 7}
\subsection{Aula 8}
\subsection{Aula 9}
\subsection{Aula 10}
