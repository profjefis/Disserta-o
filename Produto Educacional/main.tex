%%%%%%%%%%%%%%%%%%%%%%%%%%%%%%%%%%%%%%%%%
% The Legrand Orange Book
% LaTeX Template
% Version 2.1.1 (14/2/16)
%
% This template has been downloaded from:
% http://www.LaTeXTemplates.com
%
% Original author:
% Mathias Legrand (legrand.mathias@gmail.com) with modifications by:
% Vel (vel@latextemplates.com)
%
% License:
% CC BY-NC-SA 3.0 (http://creativecommons.org/licenses/by-nc-sa/3.0/)
%
% Compiling this template:
% This template uses biber for its bibliography and makeindex for its index.
% When you first open the template, compile it from the command line with the 
% commands below to make sure your LaTeX distribution is configured correctly:
%
% 1) pdflatex main
% 2) makeindex main.idx -s StyleInd.ist
% 3) biber main
% 4) pdflatex main x 2
%
% After this, when you wish to update the bibliography/index use the appropriate
% command above and make sure to compile with pdflatex several times 
% afterwards to propagate your changes to the document.
%
% This template also uses a number of packages which may need to be
% updated to the newest versions for the template to compile. It is strongly
% recommended you update your LaTeX distribution if you have any
% compilation errors.
%
% Important note:
% Chapter heading images should have a 2:1 width:height ratio,
% e.g. 920px width and 460px height.
%
%%%%%%%%%%%%%%%%%%%%%%%%%%%%%%%%%%%%%%%%%

%----------------------------------------------------------------------------------------
%	PACKAGES AND OTHER DOCUMENT CONFIGURATIONS
%----------------------------------------------------------------------------------------

\documentclass[12pt,fleqn]{book} % Default font size and left-justified equations

%----------------------------------------------------------------------------------------

\input{structure} % Insert the commands.tex file which contains the majority of the structure behind the template

\begin{document}

%----------------------------------------------------------------------------------------
%	TITLE PAGE
%----------------------------------------------------------------------------------------

\begingroup
\thispagestyle{empty}
\begin{tikzpicture}[remember picture,overlay]
\coordinate [below=12cm] (midpoint) at (current page.north);
\node at (current page.north west)
{\begin{tikzpicture}[remember picture,overlay]
\node[anchor=north west,inner sep=0pt] at (0,0) {\includegraphics[width=\paperwidth]{background}}; % Background image
\draw[anchor=north] (midpoint) node [fill=ocre!30!white,fill opacity=0.6,text opacity=1,inner sep=1cm]{\Huge\centering\bfseries\sffamily\parbox[c][][t]{\paperwidth}{\centering Jogos Digitais: Uma Abordagem de Física de Partículas Elementares no Ensino Médio\\[15pt] % Book title
{\Large MNPEF/IF-UnB}\\[20pt] % Subtitle
{\huge Prof. Jefferson Rodrigues de Oliveira  \\Profª Drª Vanessa Carvalho de Andrade}}}; % Author name
\end{tikzpicture}};
\end{tikzpicture}
\vfill
\endgroup

%----------------------------------------------------------------------------------------
%	COPYRIGHT PAGE
%----------------------------------------------------------------------------------------

\newpage
~\vfill
\thispagestyle{empty}

\noindent Copyright \copyright\ 2013 John Smith\\ % Copyright notice

\noindent \textsc{Published by Publisher}\\ % Publisher

\noindent \textsc{book-website.com}\\ % URL

\noindent Licensed under the Creative Commons Attribution-NonCommercial 3.0 Unported License (the \aspas{License}). You may not use this file except in compliance with the License. You may obtain a copy of the License at \url{http://creativecommons.org/licenses/by-nc/3.0}. Unless required by applicable law or agreed to in writing, software distributed under the License is distributed on an \textsc{``as is'' basis, without warranties or conditions of any kind}, either express or implied. See the License for the specific language governing permissions and limitations under the License.\\ % License information

\noindent \textit{First printing, March 2013} % Printing/edition date

%----------------------------------------------------------------------------------------
%	TABLE OF CONTENTS
%----------------------------------------------------------------------------------------

%\usechapterimagefalse % If you don't want to include a chapter image, use this to toggle images off - it can be enabled later with \usechapterimagetrue

\chapterimage{chapter_head_1.pdf} % Table of contents heading image

\pagestyle{empty} % No headers

\tableofcontents % Print the table of contents itself

\cleardoublepage % Forces the first chapter to start on an odd page so it's on the right

\pagestyle{fancy} % Print headers again

%----------------------------------------------------------------------------------------
%	PART
%----------------------------------------------------------------------------------------

%\part{Part One}

%----------------------------------------------------------------------------------------
%	CHAPTER 1
%----------------------------------------------------------------------------------------

\chapterimage{chapter_head_2.pdf} % Chapter heading image

\chapter{Introdução}

\lipsum[1-2] % Dummy text

\section{Inserção de Física de Partículas Elementares no Ensino Médio}\index{Paragraphs of Text}

\lipsum[1-7] % Dummy text

%------------------------------------------------

\section{Aprendizagem Baseada em Jogos Digitais}\index{Citation}

This statement requires citation \cite{book_key}; this one is more specific \cite[122]{article_key}.

\lipsum[1-7] % Dummy text
%------------------------------------------------


%----------------------------------------------------------------------------------------
%	CHAPTER 2
%----------------------------------------------------------------------------------------

\chapter{Tutorias do \textit{Scratch}}

\lipsum[1-2] % Dummy text

\section{Livros}\index{Theorems!Several Equations}

Em \fnurl{\aspas{Aprenda a Programar com Scratch}}{https://novatec.com.br/livros/aprenda-programar-com-scratch/}, o autor Majed Marji utiliza o Scratch para explicar os conceitos essenciais necessários à resolução de problemas de programação do mundo real. Os blocos nomeados e diferenciados por cores mostram claramente cada passo lógico em um dado script, e, com apenas um clique, você pode até mesmo testar qualquer parte de seu script para verificar sua lógica. Você aprenderá a:

\begin{itemize}
	\item Controlar a eficiência de laços e recursões repetitivas;
	\item Utilizar instruções if/else e operadores lógicos para tomar decisões;
	\item Armazenar dados em variáveis e listas para serem utilizados em seu programa;
	
	\item Ler, armazenar e manipular dados de entrada dos usuários;
	\item Implementar algoritmos fundamentais da ciência da computação, como pesquisas lineares e bubble sorts.
\end{itemize}

\section{Vídeos}\index{Theorems!Several Equations}

O canal do You Tube {\aspas{A pensar em...}, apresenta uma playlist de 18 vídeos sobre scratch, intitulada de: \fnurl{\aspas{A Pensar em... Scratch (Tutorial em Português)}}{https://goo.gl/vwHcCR}, esta série de vídeos são apresentados os seguintes temas: o que é Scratch, comunidade Scratch, aplicação offline (versão 1.4) e utilização de blocos.
	
\aspas{ILUSTRADUCAS} é um canal de vídeo-aulas com dicas de Ilustração, Pintura Digital, animação 3d e Arte para Games. Este canal apresenta uma série de 10 vídeos intitulada \fnurl{\aspas{SCRATCH: Criando um jogo do zero!}}{https://goo.gl/eK3fuZ}, esta série de vídeos ensina a programar de forma simples e criar um jogo em Scratch. 
	
\fnurl{\aspas{Curso completo de Scratch}}{https://goo.gl/Yey2rW}, curso de Scratch oferecido pelo Programa NERDS (Núcleo Educacional de Robótica e Desenvolvimento de Software) da Fronteira e Programa PET (Programa de Educação Tutorial) da Fronteira da Universidade Federal de Mato Grosso do Sul (UFMS) câmpus Ponta Porã. Este curso tem como objetivo a formação de professores para uso de novas tecnologias na sala de aula. O curso é ministrado pela Esteice Janaina. 27 vídeos.
	
\fnurl{\aspas{Curso de Scratch - Programação e lógica para Kids e professores}}{https://goo.gl/XHVAjJ} do canal \aspas{Curso de Excel Online}, curso básico e gratuito. 13 vídeos. 
	
\fnurl{\aspas{Scratch Programming}}{https://goo.gl/mxTUZc} inglês do canal \aspas{Blank Editor} com 20 vídeos.

\section{Artigos}\index{Theorems!Several Equations}

\fnurl{\aspas{Scratch Unicamp}}{https://www.ft.unicamp.br/liag/scratch/category/downloadstutoriais/}, do site "Raciocinando E<>! Blocos" é um espaço para demonstrar a pesquisa, o desenvolvimento de produtos e processos voltados a atividades de Aprendizagem Criativa. Inclui pesquisas de Graduação, Mestrado e Doutorado realizadas no escopo do Grupo do LIAG (Laboratório de Informática Aprendizagem e Gestão) da Faculdade de Tecnologia da UNICAMP. Os integrantes do LIAG promovem também atividades de extensão destas pesquisas para a sociedade, em especial para o público escolar.

Os projetos, artigos e experiências do grupo encontram-se no site com o intuito de divulgar os trabalhos que estão sendo feitos em Aprendizagem Criativa, o site trás também noticias, novidades e o que está acontecendo nas escolas quando o tema é Aprendizagem Criativa.

O site \fnurl{\aspas{Scratch Brasil}}{http://www.scratchbrasil.net.br/index.php/materiais/tutoriais.html}, fornece material gratuito em língua portuguesa sobre a ferramenta, além de mostrar notícias, eventos, tutoriais, vídeo aulas, entre outras informações de como professores e alunos podem usar a plataforma em sala de aula para a criação de jogos e animações com temas educativos.

Além disso, o Scratch Brasil realiza oficinas, palestras e demais eventos voltados para a plataforma Scratch.

\fnurl{\aspas{EduScratch}}{http://projectos.ese.ips.pt/eduscratch/index.php/36-uncategorised/110-artigos-sobre-o-scratch} é um projeto que visa promover a utilização educativa do Scratch através do apoio, formação e partilha de experiências na comunidade educativa. O site tem uma lista de vários artigos. 

%----------------------------------------------------------------------------------------
%	PART
%----------------------------------------------------------------------------------------

%\part{Part Two}

%----------------------------------------------------------------------------------------
%	CHAPTER 3
%----------------------------------------------------------------------------------------

\chapterimage{chapter_head_1.pdf} % Chapter heading image

\lipsum[1-2] % Dummy text

\chapter{O Jogo \aspas{Em Busca do Bóson de Higgs}}
\lipsum[1-7] % Dummy text

\section{Acesso ao Jogo}
\lipsum[1-7] % Dummy text

\section{Atividades Extras}
\lipsum[1-7] % Dummy text

\section{O Jogo principal}
\lipsum[1-7] % Dummy text

\section{A fase Final}
\lipsum[1-7] % Dummy text

%----------------------------------------------------------------------------------------
%	BIBLIOGRAPHY
%----------------------------------------------------------------------------------------

\chapter*{Referências}
\addcontentsline{toc}{chapter}{\textcolor{ocre}{Referências}}
\section*{Livros}
\addcontentsline{toc}{section}{Livros}
\printbibliography[heading=bibempty,type=book]
\section*{Artigos}
\addcontentsline{toc}{section}{Artigos}
\printbibliography[heading=bibempty,type=article]

%----------------------------------------------------------------------------------------
%	INDEX
%----------------------------------------------------------------------------------------

\cleardoublepage
\phantomsection
\setlength{\columnsep}{0.75cm}
\addcontentsline{toc}{chapter}{\textcolor{ocre}{Index}}
\printindex

%----------------------------------------------------------------------------------------

\end{document}
