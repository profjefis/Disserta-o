%%%%%%%%%%%%%%%%%%%%%%%%%%%%%%%%%%%%%%%%%
% The Legrand Orange Book
% LaTeX Template
% Version 2.1.1 (14/2/16)
%
% This template has been downloaded from:
% http://www.LaTeXTemplates.com
%
% Original author:
% Mathias Legrand (legrand.mathias@gmail.com) with modifications by:
% Vel (vel@latextemplates.com)
%
% License:
% CC BY-NC-SA 3.0 (http://creativecommons.org/licenses/by-nc-sa/3.0/)
%
% Compiling this template:
% This template uses biber for its bibliography and makeindex for its index.
% When you first open the template, compile it from the command line with the 
% commands below to make sure your LaTeX distribution is configured correctly:
%
% 1) pdflatex main
% 2) makeindex main.idx -s StyleInd.ist
% 3) biber main
% 4) pdflatex main x 2
%
% After this, when you wish to update the bibliography/index use the appropriate
% command above and make sure to compile with pdflatex several times 
% afterwards to propagate your changes to the document.
%
% This template also uses a number of packages which may need to be
% updated to the newest versions for the template to compile. It is strongly
% recommended you update your LaTeX distribution if you have any
% compilation errors.
%
% Important note:
% Chapter heading images should have a 2:1 width:height ratio,
% e.g. 920px width and 460px height.
%
%%%%%%%%%%%%%%%%%%%%%%%%%%%%%%%%%%%%%%%%%

%----------------------------------------------------------------------------------------
%	PACKAGES AND OTHER DOCUMENT CONfigureTIONS
%----------------------------------------------------------------------------------------

\documentclass[12pt,fleqn]{book} % Default font size and left-justified equations
\usepackage{enumerate}
\usepackage{quoting}
\usepackage{setspace}

%----------------------------------------------------------------------------------------

\input{structure} % Insert the commands.tex file which contains the majority of the structure behind the template
\onehalfspacing
\begin{document}

%----------------------------------------------------------------------------------------
%	TITLE PAGE
%----------------------------------------------------------------------------------------

\begingroup
\thispagestyle{empty}
\begin{tikzpicture}[remember picture,overlay]
\coordinate [below=12cm] (midpoint) at (current page.north);
\node at (current page.north west)
{\begin{tikzpicture}[remember picture,overlay]
\node[anchor=north west,inner sep=0pt] at (0,0) {\includegraphics[width=\paperwidth]{capa_2}}; % Background image
\draw[anchor=north] (midpoint) node [fill=ocre!30!white,fill opacity=0.6,text opacity=1,inner sep=1cm]{\Huge\centering\bfseries\sffamily\parbox[c][][t]{\paperwidth}{\centering Jogos Digitais: Uma Abordagem de Física de Partículas Elementares no Ensino Médio\\[15pt] % Book title
{\Large MNPEF/IF-UnB}\\[20pt] % Subtitle
{\huge Jefferson Rodrigues de Oliveira  \\Profª Drª Vanessa Carvalho de Andrade}}}; % Author name
\end{tikzpicture}};
\end{tikzpicture}
\vfill
\endgroup

%----------------------------------------------------------------------------------------
%	COPYRIGHT PAGE
%----------------------------------------------------------------------------------------

\newpage
~\vfill
\thispagestyle{empty}

%\noindent Copyright \copyright\ 2013 John Smith\\ % Copyright notice

%\noindent \textsc{Published by Publisher}\\ % Publisher

%\noindent \textsc{book-website.com}\\ % URL

\noindent O template original deste ebook \aspas{The Legrand Orange Book Template (English)} cujo author é denominado de Mathias Legrand, possui licença livre para modificar e compartilhar de forma não comercial \href{https://creativecommons.org/licenses/by-nc-sa/3.0/}{CC BY-NC-SA 3.0}. As Imagens dos capítulos foram retiradas de sites que permitiam reutilização com modificação: \href{https://pixabay.com/}{Pixabay} e \href{https://www.wikimedia.org/}{Wikimedia}.\\ % License information

\noindent \textit{Junho de 2018} % Printing/edition date

%----------------------------------------------------------------------------------------
%	TABLE OF CONTENTS
%----------------------------------------------------------------------------------------

%\usechapterimagefalse % If you don't want to include a chapter image, use this to toggle images off - it can be enabled later with \usechapterimagetrue

\chapterimage{sumario.pdf} % Table of contents heading image

\pagestyle{empty} % No headers

\tableofcontents % Print the table of contents itself

\cleardoublepage % Forces the first chapter to start on an odd page so it's on the right

\pagestyle{fancy} % Print headers again

%----------------------------------------------------------------------------------------
%	PART
%----------------------------------------------------------------------------------------

%\part{Part One}

%----------------------------------------------------------------------------------------
%	CHAPTER 1
%----------------------------------------------------------------------------------------

\chapterimage{game.pdf} % Chapter heading image

\chapter{Introdução}
Vivemos em um mundo onde é notório a magnitude de alcance das tecnologias digitais. Os jovens de hoje, conhecidos como \aspas{nativos digitais}, possuem um comportamento completamente diferente dos jovens que nasceram a partir da metade do século passado, conhecidos como \aspas{imigrantes digitais}. Educar uma nova geração por meio de métodos antigos utilizando ferramentas que se tornaram arcaicas, são ineficientes. Acrescentar diversão ao processo não apenas fará que a aprendizagem se tornem muito mais agradáveis e envolventes, mas também os tornará muito mais eficazes \cite{prensky2012aprendizagem}.

Tendo esta motivação inicial e percebendo a escassez de materiais relacionados á Física de Partículas Elementares para o Ensino Médio \cite{siqueira2005revisando}, decidimos elaborar um jogo com linguagem simples e acessível para que o alunos seja instigado e se sinta motivado para aprender sobre alguns conceitos da física de partículas.

Por ser uma linguagem mais simples e intuitiva, escolhemos trabalhar com o \textit{Scratch}. No próximo capítulo, relataremos as principais características desta ferramenta e indicaremos materiais de estudo para você ficar afiado na programação de jogos digitais. 


%------------------------------------------------


%----------------------------------------------------------------------------------------
%	CHAPTER 2
%----------------------------------------------------------------------------------------


\chapterimage{scratch.pdf} % Chapter heading image

\chapter{Tutorias do \textit{Scratch}}

Segundo o site \textcite{on:scratch2018}, o \textit{Scratch} é um projeto do \textit{Lifelong Kindergarten Group} do MIT \textit{Media Lab}. Disponibilizado gratuitamente, o \textit{Scratch} ajuda os jovens a pensar de forma criativa, a raciocinar sistematicamente e a trabalhar colaborativamente competências essenciais à vida no século XXI. Com ele é possível programar suas próprias histórias, jogos e animações interativas, além de poder compartilhar suas criações com toda a comunidade. Justamente esta comunidade ativa, que torna-o bastante interativo, pois os estudantes podem compartilhar seus projetos e aprender uns com os outros \apud{resnick2009scratch}{bastos2010schatch}.

O Termo \textit{Scratch} tem origem da técnica de \textit{scratching} utilizadas pelos DJs (\textit{disc jockeys}) do \textit{hip-hop}. Pois é possível fazer algo semelhante com o \textit{Scratch}, que nos permite controlar ações e interações entre diferentes tipos de imagens, sons e cores, misturando-os de forma criativa \cite{marques2009recuperar}.

\textapud{klopfer2004programming}{marques2009recuperar}, cita os principais aspectos-chave inovadores do Scratch:

\begin{enumerate}[a)]
	\item Programação com blocos-de-construção (\textit{building-blocks}) - Para escrever programas em \textit{Scratch}, encaixam-se blocos gráficos uns nos outros, formando empilhamentos ordenados (\textit{stacks}). Os blocos são concebidos para se poderem encaixar apenas de forma que faça sentido sintaticamente, não ocorrendo, assim, erros de sintaxe. As sequências de instruções podem ser modificadas mesmo com o programa a correr, o que facilita a experimentação simples de novas ideias e o multiprocessamento é integrado de forma simples podendo ser executadas instruções paralelamente por diferentes conjuntos de blocos;
	\item Manipulação de media - O \textit{Scratch} permite a construção de programas que controlam e misturam gráficos, animação, texto, música e som. Amplia as atividades de manipulação de media que são populares na cultura atual;
	\item Partilha e colaboração - A página de Internet do \textit{Scratch} fornece inspiração e audiência: podemos experimentar os projetos de outros, reutilizar e adaptar as suas imagens e \textit{scripts}, e divulgar os nossos próprios projetos. A meta final é desenvolver uma comunidade e uma cultura de partilha em torno do \textit{Scratch};
	\item Opção de múltiplas línguas, incluindo a portuguesa, desde a sua concepção - Pretende promover a criação de uma cultura \textit{Scratch} na comunidade internacional.	
\end{enumerate}

Estes aspectos inovadores trazem uma aprendizagem muito mais fácil e ativa. Destacamos os principais motivos que fizeram do \textit{Scratch} nossa escolha:

\begin{enumerate}[i)]
	\item Facilidade de aprendizagem: ferramenta intuitiva e lúdica; 
	\item Programação em blocos: programação de \aspas{encaixe}\footnote{Como se fossem peças de LEGO}, evitando possíveis erros de sintaxe.
	\item Colaboração: comunidade ativa e interativa com fóruns especializados. 
\end{enumerate}

A seguir, indicaremos alguns livros, vídeos e sites sobre o \textit{Scratch}.

\section{Livros}

\textcite{marji2014aprenda} utiliza o \textit{Scratch} para explicar os conceitos essenciais necessários à resolução de problemas de programação do mundo real. Os blocos nomeados e diferenciados por cores mostram claramente cada passo lógico em um dado \textit{script}, e, com apenas um clique, você pode até mesmo testar qualquer parte de seu \textit{script} para verificar sua lógica. Você aprenderá a:

\begin{itemize}
	\item Controlar a eficiência de laços e recursões repetitivas;
	\item Utilizar instruções \textit{if/else} e operadores lógicos para tomar decisões;
	\item Armazenar dados em variáveis e listas para serem utilizados em seu programa;
	
	\item Ler, armazenar e manipular dados de entrada dos usuários;
	\item Implementar algoritmos fundamentais da ciência da computação, como pesquisas lineares e \textit{bubble sorts}.
\end{itemize}

\section{Vídeos}

O canal do You Tube \textcite{apensarem2018} apresenta uma \textit{playlist} de 18 vídeos sobre Scratch, nesta série são apresentados os seguintes temas: o que é \textit{Scratch}, comunidade \textit{Scratch}, aplicação \textit{offline} (versão 1.4) e utilização de blocos.
	
\textcite{ilustradicas2018} é um canal de vídeo-aulas com dicas de ilustração, pintura digital, animação 3d e Arte para Games. Este canal apresenta uma série de 10 vídeos, esta série ensina a programar de forma simples e criar um jogo em \textit{Scratch}. 
	
\textcite{cursocompleto2018} é um curso de Scratch oferecido pelo Programa NERDS (Núcleo Educacional de Robótica e Desenvolvimento de Software) da Fronteira e Programa PET (Programa de Educação Tutorial) da Fronteira da Universidade Federal de Mato Grosso do Sul (UFMS) campus Ponta Porã. Este curso tem como objetivo a formação de professores para uso de novas tecnologias na sala de aula. O curso é ministrado pela Esteice Janaina e possui 27 vídeos.
	
\textcite{cursoexcel2018} é curso básico e gratuito sobre \textit{Scratch} que possui 13 vídeos. 
	
\textcite{blank2018} é um canal de língua inglesa que possui 20 vídeos.

\section{Artigos}\index{Theorems!Several Equations}

\textcite{liag2018} é um espaço para demonstrar a pesquisa, o desenvolvimento de produtos e processos voltados a atividades de Aprendizagem Criativa. Inclui pesquisas de Graduação, Mestrado e Doutorado realizadas no escopo do Grupo do LIAG (Laboratório de Informática Aprendizagem e Gestão) da Faculdade de Tecnologia da UNICAMP. Os integrantes do LIAG promovem também atividades de extensão destas pesquisas para a sociedade, em especial para o público escolar.

Os projetos, artigos e experiências do grupo encontram-se no site com o intuito de divulgar os trabalhos que estão sendo feitos em Aprendizagem Criativa, o site trás também noticias, novidades e o que está acontecendo nas escolas quando o tema é Aprendizagem Criativa.

\textcite{scratchbrasil2018} fornece material gratuito em língua portuguesa sobre a ferramenta, além de mostrar notícias, eventos, tutoriais, vídeo aulas, entre outras informações de como professores e alunos podem usar a plataforma em sala de aula para a criação de jogos e animações com temas educativos.

Além disso, o \textit{Scratch} Brasil realiza oficinas, palestras e demais eventos voltados para a plataforma \textit{Scratch}.

\textcite{eduscratch2018} é um projeto que visa promover a utilização educativa do \textit{Scratch} por meio do apoio, formação e partilha de experiências na comunidade educativa. O site tem uma lista de vários artigos. 

%----------------------------------------------------------------------------------------
%	PART
%----------------------------------------------------------------------------------------

%\part{Part Two}

%----------------------------------------------------------------------------------------
%	CHAPTER 3
%----------------------------------------------------------------------------------------

\chapterimage{lhc.pdf} % Chapter heading image

\chapter{O Jogo \aspas{Em Busca do Bóson de Higgs}}
\section{introdução}
Olá a todos os entusiastas dos games! Vamos explicar para vocês o passo-a-passo de como funciona a dinâmica do jogo \aspas{Em Busca do Bóson de Higgs}.

Primeiramente, vamos ao título, exatamente o porquê do título do jogo. Toda a temática da nossa sequência didática gira em torno da física de partículas elementares, o bóson de Higgs juntamente com o LHC são os assuntos que mais foram noticiados na mídia em uma forma geral, e é exatamente esses assuntos que os estudantes mais fazem perguntas e que mais despertam interesse neles quando se fala em física de partículas.

O jogo consiste basicamente em um personagem (\textit{hero}) que está em busca de entender o funcionamento da detecção do bóson de Higgs.

Entretanto, para entender o funcionamento deste detecção, o personagem tem que dialogar com várias autoridades \footnote{Peter Higgs, Linus Pauling, César Lattes e Albert Einstein} da física e da química que darão todo o arcabouço teórico para esse entendimento. Vale destacar aqui a importância destes diálogos, pois contém várias informações históricas relevantes sobre estas personalidades. 

O jogo tem um formato de \textit{quiz}, no qual as perguntas podem ser respondidas com auxilio do diálogo.

\section{Acesso ao Jogo}
É possível acessar o jogo de duas formas: 

Forma Indireta:

1º Passo: Entrar no site do \href{https://scratch.mit.edu}{\textit{Scratch}}: \url{https://scratch.mit.edu}.

\begin{figure}[ht]
	\centering
	\includegraphics[width=0.7 \textwidth]{Produto/site_scratch}
	\caption{Site \textit{Scratch}}
	\label{fig:app_a:sitescratch}
\end{figure}

2º Passo: Ir na barra de pesquisa e digitar o nome do jogo \aspas{Em Busca do Bóson de Higgs}. 

\begin{figure}[h]
	\centering
	\includegraphics[width=0.7 \textwidth]{Produto/pesquisa}
	\caption{Pesquisa sobre o jogo}
	\label{fig:app_pesquisa}
\end{figure}

\newpage

Clicar sobre o projeto que aparece na tela.

\begin{figure}[h]
	\centering
	\includegraphics[width=0.7 \textwidth]{Produto/site_jogo}
	\caption{O projeto}
	\label{fig:app_a:projeto}
\end{figure}

Forma Direta:

É possível acessar a mesma tela inicial do jogo acessando diretamente o seguinte link \url{https://scratch.mit.edu/projects/171396087/}.

Enfim, ao iniciar o projeto do jogo é necessário clicar no ícone da bandeira verde à direita, enquanto que, para jogar em formato de tela cheia (caso queira), clique no ícone retangular azul à esquerda.

Seguindo estes passos, é possível obter o seguinte resultado:

\begin{figure}[h]
	\centering
	\includegraphics[width=0.65 \textwidth]{Produto/tela_inicial}
	\caption{A tela inicial}
	\label{fig:app_a:telainicial}
\end{figure}

\newpage

Clicando no botão \aspas{Cliqe Aqui!}, aparecerá a seguinte tela:

\begin{figure}[h]
	\centering
	\includegraphics[width=0.65 \textwidth]{Produto/options}
	\caption{Opções}
	\label{fig:app_a:options}
\end{figure}


Antes de comentar sobre o jogo em si, vamos primeiro relatar sobre os créditos e sobre os bônus.

Créditos:

Os créditos retratam toda as referências que utilizamos para elaborar o jogo.

\begin{figure}[h]
	\centering
	\includegraphics[width=0.65 \textwidth]{Produto/credits}
	\caption{Créditos}
	\label{fig:app_a:credits}
\end{figure}

O créditos estão relacionados com: a programação, o enredo, os \textit{backgrounds} (panos de fundo), os \textit{sprites} (imagens dos personagens, cenários,...) e os efeitos sonoros.

\newpage

\section{Atividades Extras}
Existem dois tipos de bônus (atividades extras):

\begin{figure}[h]
	\centering
	\includegraphics[width=0.65 \textwidth]{Produto/extras}
	\caption{Atividades extras}
	\label{fig:app_a:extras}
\end{figure}


1º Classificação das Partículas

Clicando no botão \aspas{Classificação das Partículas}, aparecerá uma tela com a explicação da dinâmica do mini-game.

\begin{figure}[h]
	\centering
	\includegraphics[width=0.65 \textwidth]{Produto/class1}
	\caption{Classificação: tela inicial}
	\label{fig:app_a:class1}
\end{figure}

\newpage

Após a explicação, aparecerá uma tela com a tabela de classificação das partículas elementares do modelo padrão.

\begin{figure}[h]
	\centering
	\includegraphics[width=0.65 \textwidth]{Produto/class2}
	\caption{Partículas do Modelo Padrão}
	\label{fig:app_a:class2}
\end{figure}

No lado direito, aparecerá uma contagem regressiva de 10 segundos. Quando esta contagem chega a 0, a tabela some.

Após esta contagem regressiva, aparecerá uma tela de inicio da classificação, na parte de baixo uma seta e a partícula a ser classificada e na parte de cima 3 blocos em que as partículas deverão ser classificadas (Quarks, Férmions e Bósons).

\begin{figure}[h]
	\centering
	\includegraphics[width=0.63 \textwidth]{Produto/class3}
	\caption{Classe de partículas}
	\label{fig:app_a:class3}
\end{figure}

\newpage

Caso a classificação esteja de acordo com o modelo padrão, a pontuação aumenta em um ponto.

\begin{figure}[h]
	\centering
	\includegraphics[width=0.65 \textwidth]{Produto/class5}
	\caption{Acertando a classificação}
	\label{fig:app_a:class5}
\end{figure}


O objetivo do mini-game é classificar adequadamente 10 partículas elementares do modelo padrão. Caso logre êxito, aparecerá a seguinte tela:

\begin{figure}[h]
	\centering
	\includegraphics[width=0.63 \textwidth]{Produto/class10}
	\caption{Alcançando o objetivo}
	\label{fig:app_a:class10}
\end{figure}

\newpage

Entretanto, caso a classificação não esteja de acordo com o modelo padrão, os blocos descem...

\begin{figure}[h]
	\centering
	\includegraphics[width=0.65 \textwidth]{Produto/class4}
	\caption{Blocos descendo}
	\label{fig:app_a:class4}
\end{figure}

Caso algum dos blocos colidir com a seta, o jogo terminará:

\begin{figure}[h]
	\centering
	\includegraphics[width=0.63 \textwidth]{Produto/class7}
	\caption{Fim de jogo}
	\label{fig:app_a:class7}
\end{figure}

\newpage

Independente se conseguir alcançar o objetivo ou não, você poderá reiniciar e jogar novamente, quantas vezes desejar.

\begin{figure}[h]
	\centering
	\includegraphics[width=0.65 \textwidth]{Produto/class9}
	\caption{Possível reinicio}
	\label{fig:app_a:class9}
\end{figure}


Mesmo que este mini-game esteja relacionado com conhecimentos memorísticos e que tenha traços de aprendizagem mecânica, ele tem como objetivo não apenas o \aspas{decoreba} por si mesmo, mas sim, uma ambientação dos termos mais complexos da FPE.

2º Simulação relativística

Esta simulação de relatividade restrita tem como objetivo ilustrar o efeito relativístico de dilatação do tempo da partícula denominada: múon.

\begin{figure}[h]
	\centering
	\includegraphics[width=0.6 \textwidth]{Produto/sim1}
	\caption{Explicação da simulação}
	\label{fig:app_a:sim1}
\end{figure}

A principal ideia desta simulação consiste em controlar o parâmetro velocidade do múon e verificar o que acontece com o tempo de vida médio medido no laboratório.

\begin{figure}[h]
	\centering
	\includegraphics[width=0.6 \textwidth]{Produto/sim2}
	\caption{A simulação}
	\label{fig:app_a:sim2}
\end{figure}


\section{O Jogo Principal}
O jogo \aspas{Em busca do Bóson de Higgs}

Você inicia o jogo no lado norte do mapa, logo abaixo de uma casa. 

\begin{figure}[h]
	\centering
	\includegraphics[width=0.65 \textwidth]{Produto/jogo_1}
	\caption{Início}
	\label{fig:app_a:jogo1}
\end{figure}

\newpage

Aparecerá algumas instruções, você deverá permanecer parado no intuito de ler as instruções. Apenas quando concluí-las, você poderá desbravar o mapa da cidade. 

\begin{figure}[h]
	\centering
	\includegraphics[width=0.65 \textwidth]{Produto/jogo_2}
	\caption{Instruções Inicias}
	\label{fig:app_a:jogo2}
\end{figure}

Após as instruções, procure um dos principais personagens da narrativa, Peter Higgs.

\begin{figure}[h]
	\centering
	\includegraphics[width=0.65 \textwidth]{Produto/jogo_3}
	\caption{Diálogo com Higgs 1}
	\label{fig:app_a:jogo3}
\end{figure}

\newpage

Quando encontrar o personagem, fique de frente a ele e tecle \aspas{espaço}, Higgs irá perguntar seu nome\footnote{Ao longo do jogo, todos os personagens irá interagir com você a partir do seu nome}, neste momento você deverá escrever seu nome via teclado e apertar a tecla \aspas{enter}.

\begin{figure}[h]
	\centering
	\includegraphics[width=0.65 \textwidth]{Produto/jogo_4}
	\caption{Diálogo com Higgs 2}
	\label{fig:app_a:jogo4}
\end{figure}

Após o diálogo inicial com Higgs, ele mostrará a indicação do próximo passo para concluir a missão.

\begin{figure}[h]
	\centering
	\includegraphics[width=0.65 \textwidth]{Produto/jogo_5}
	\caption{Diálogo com Higgs 3}
	\label{fig:app_a:jogo5}
\end{figure}

\newpage

Ao encontrar com o próximo personagem, Linus Pauling, aperte a tecle novamente \aspas{espaço} para inicial do diálogo.

\begin{figure}[h]
	\centering
	\includegraphics[width=0.65 \textwidth]{Produto/jogo_6}
	\caption{Diálogo com Pauling 1}
	\label{fig:app_a:jogo6}
\end{figure}

Aqui começa as ideias introdutórias da física de partículas elementares.

\begin{figure}[h]
	\centering
	\includegraphics[width=0.65 \textwidth]{Produto/jogo_7}
	\caption{Diálogo com Pauling 2}
	\label{fig:app_a:jogo7}
\end{figure}

\newpage

Após o diálogo, Pauling avisará para procurar o físico César Lattes.

\begin{figure}[h]
	\centering
	\includegraphics[width=0.65 \textwidth]{Produto/jogo_8}
	\caption{Diálogo com Pauling 3}
	\label{fig:app_a:jogo8}
\end{figure}

Aqui começa o diálogo com Lattes.

\begin{figure}[h]
	\centering
	\includegraphics[width=0.65 \textwidth]{Produto/jogo_9}
	\caption{Diálogo com Lattes 1}
	\label{fig:app_a:jogo9}
\end{figure}

\newpage

Após os diálogos, são realizadas o \textit{quiz} referentes ao diálogo dos personagens.

\begin{figure}[h]
	\centering
	\includegraphics[width=0.65 \textwidth]{Produto/jogo_10}
	\caption{Diálogo com Lattes 2}
	\label{fig:app_a:jogo10}
\end{figure}

Conforme acerte as perguntas, seus pontos aumentam.

\begin{figure}[h]
	\centering
	\includegraphics[width=0.65 \textwidth]{Produto/jogo_11}
	\caption{Diálogo com Lattes 3}
	\label{fig:app_a:jogo11}
\end{figure}

\newpage

Caso erre, você deixa de ganhar a pontuação referente à pergunta.

\begin{figure}[h]
	\centering
	\includegraphics[width=0.65 \textwidth]{Produto/jogo_12}
	\caption{Diálogo com Lattes 3}
	\label{fig:app_a:jogo12}
\end{figure}

Aqui um exemplo de pergunta sobre o modelo padrão.

\begin{figure}[h]
	\centering
	\includegraphics[width=0.65 \textwidth]{Produto/jogo_13}
	\caption{Diálogo com Lattes 4}
	\label{fig:app_a:jogo13}
\end{figure}

\newpage

Lattes relata para que o jogador procure Einstein.

\begin{figure}[h]
	\centering
	\includegraphics[width=0.65 \textwidth]{Produto/jogo_14}
	\caption{Diálogo com Lattes 5}
	\label{fig:app_a:jogo14}
\end{figure}

Após o encontro com Einstein, inicia-se o diálogo.

\begin{figure}[h]
	\centering
	\includegraphics[width=0.65 \textwidth]{Produto/jogo_15}
	\caption{Diálogo com Einstein 1}
	\label{fig:app_a:jogo15}
\end{figure}

\newpage

Após alguns diálogo, Einstein realiza algumas perguntas sobre interações fundamentais.

\begin{figure}[h]
	\centering
	\includegraphics[width=0.65 \textwidth]{Produto/jogo_16}
	\caption{Diálogo com Einstein 2}
	\label{fig:app_a:jogo16}
\end{figure}


Após o diálogo com Einstein, o personagem deverá encontrar novamente com Peter Higgs, que fará mais algumas perguntas...

\begin{figure}[h]
	\centering
	\includegraphics[width=0.65 \textwidth]{Produto/jogo_17}
	\caption{Diálogo final com Higgs 1}
	\label{fig:app_a:jogo17}
\end{figure}

\newpage

Aqui, Higgs explica sobre a última missão...

\begin{figure}[h]
	\centering
	\includegraphics[width=0.65 \textwidth]{Produto/jogo_18}
	\caption{Diálogo final com Higgs 2}
	\label{fig:app_a:jogo18}
\end{figure}

Por fim, chegamos na reta final...

\begin{figure}[h]
	\centering
	\includegraphics[width=0.65 \textwidth]{Produto/jogo_19}
	\caption{Diálogo final com Higgs 3}
	\label{fig:app_a:jogo19}
\end{figure}

\newpage

\section{A Missão Final}

Inicia-se a simulação da colisão de hádrons.

\begin{figure}[h]
	\centering
	\includegraphics[width=0.65 \textwidth]{Produto/final1}
	\caption{Fase final - Parte 1}
	\label{fig:app_a:final1}
\end{figure}

Nesta etapa, você perceberá que irá descer várias partículas, dentre elas: o fóton, o elétron e o próton.

\begin{figure}[h]
	\centering
	\includegraphics[width=0.65 \textwidth]{Produto/final2}
	\caption{Fase final - Parte 2}
	\label{fig:app_a:final2}
\end{figure}

\newpage

O objetivo desta última missão é a colisão do próton (que está com o personagem principal) com os prótons que descem sobre a tela.

Para cada acerto, o parâmetro de energia aumenta 25 Gev, enquanto que se você colidir as partículas de forma inadequada ou se alguma partícula colidir em você, perderás uma vida.

\begin{figure}[h]
	\centering
	\includegraphics[width=0.65 \textwidth]{Produto/final3}
	\caption{Fase final - Parte 3}
	\label{fig:app_a:final3}
\end{figure}


Caso consiga chegar em nível de energia de 125 Gev, você alcançará o objetivo e detectará o bóson de Higgs.

\begin{figure}[h]
	\centering
	\includegraphics[width=0.65 \textwidth]{Produto/final4}
	\caption{Fase final - Parte 4}
	\label{fig:app_a:final4}
\end{figure}




%----------------------------------------------------------------------------------------
%	BIBLIOGRAPHY
%----------------------------------------------------------------------------------------
\chapterimage{sumario.pdf} % Chapter heading image
\chapter*{Referências}
\addcontentsline{toc}{chapter}{\textcolor{ocre}{Referências}}
%\printbibliography[heading=bibempty]
\section*{Artigos}
%\addcontentsline{toc}{section}{Artigos}
\printbibliography[heading=bibempty,type=article]
\section*{Livros}
%\addcontentsline{toc}{section}{Livros}
\printbibliography[heading=bibempty,type=book]
\section*{Sites com artigos sobre o \textit{Scratch}}
%\addcontentsline{toc}{section}{Sites com artigos sobre o \textit{Scratch}}
\printbibliography[heading=bibempty,keyword=site]
\section*{Vídeos sobre o \textit{Scratch}}
%\addcontentsline{toc}{section}{Vídeos sobre o \textit{Scratch}}
\printbibliography[heading=bibempty,keyword=video]
\section*{Outros}
%\addcontentsline{toc}{section}{Outros}
\printbibliography[heading=bibempty,keyword=others]


%----------------------------------------------------------------------------------------
%	INDEX
%----------------------------------------------------------------------------------------

\cleardoublepage
\phantomsection
\setlength{\columnsep}{0.75cm}
\addcontentsline{toc}{chapter}{\textcolor{ocre}{Index}}
\printindex

%----------------------------------------------------------------------------------------

\end{document}
