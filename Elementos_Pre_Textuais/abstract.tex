\begin{singlespace}
Due to the shortage of didactic materials aimed at the theme of Modern and Contemporary Physics for High School, mainly non-speakers in Elementary Particle Physics, we realized the need to elaborate a sequence of techniques with an Information and Communication Technologies ( TICs)), specifically, in English learning. We also notice data generation times (known by digital natives), live in a highly technological reality. In relation to digital games, they are accessible in several ways: smartphone, tablet, notebook, console... In addition to playing several styles: RPG (role-playing game), adventure, sports, strategy... For being a language the next student, accredited for learning, are according to current generation needs and motivate because they are fun. To base our work, we use two theoretical references. The first is Ausubel, because meaningful learning permeates our whole didactic sequence. While the second is Marc Prensky, for being one of the great enthusiasts and promoters of digital games as a learning tool. Developed in Scratch (programming language and interactive online community), the game has an RPG style (type of game in which players assume roles of people) and a dynamic real-time (set of questions and answers) about the concepts of Elementary Particle Physics. We formulated a didactic sequence with a total of 10 lessons (each lasting 45 minutes). The digital game was inserted in two different moments, the first one (beginning of the sequence), served as a motivating tool for the learning, and the second (end of the sequence), served as an opportunity to fix the content and to improve the performance of the previous application . We apply this methodology in a public school of the Federal District located in the satellite city of Ceil�ndia. Our target audience and students in the third grade of regular high school, broadband and between 17 and 19 years. We note in this experience positive and negative points. While some students became more focused, motivated, and held collaborative groups to answer as questions, others were scattered because they did not like games or felt unmotivated because of technical computer problems. Finally, we believe that it is a business experience, it is necessary to consider the needs and the learning styles of the generation and of the future, besides being motivating because it is fun.
\end{singlespace}

\textbf{Keywords}: Physics Teaching, High School, Particle Physics, Learning, Digital Games.
