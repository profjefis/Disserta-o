\begin{singlespace}
\textit{Due to the shortage of teaching materials aimed at the teaching of contemporary modern physics for high school students, specially regarding Elementary Particle Physics (EPP), we understand that coming up with a didactic sequence, with some help from Communication and Information Technologies, would be necessary ? specifically for learning based on digital games. We also realized that today's students, known as digital natives, live in a heavily technological world. In terms of digital games, such students have access to them in many different ways, such as: by means of smartphones, tablets, notebook computers, video game consoles etc ? besides playing styles as varied as Role-Playing Games (RPG), adventure, sports video-games, strategy video-games and so on. Because it is a very accessible format for the students, we believe that digital games i) increase engagement for learning ii) are adequate for today's generation needs and iii) motivate students because they are fun to play. Due to the nature of our work, we consider it adequate to base it upon three theoretical references. The first one is the work of David Ausubel, because the concept of meaningful learning permeates our didactic sequence. The second one is Marco Ant�nio Moreira, due to his important contributions in the theory of meaningful learning. Last but not least, is Marc Prensky due to him being one of the great enthusiasts and promoters of digital games as tools for learning. We have thus developed a game using Scratch (a programming language and interactive on-line community). Our game is a RPG based on a series of quizzes about EPP concepts. We developed a didactic sequence with a total of ten lesson (each one last 45 minutes). This digital game was thus used in two situations: firstly at the beginning of the didactic sequence, as a motivational tool for learning and finally at the end of the sequence for reviewing purposes. This methodology was applied at a public school in Ceil�ndia, Distrito Federal (Brazilian Federal District). Our target were high school students aged 17-19 (third year of Brazilian High School). Even though there was some rejection from a small minority of students, we concluded that learning based on digital games has enormous potential to stimulate students by teaching them in a completely different manner. Finally, we believe that learning based on digital games is in accordance with this generation (and probably future generations as well) needs and learning styles ? besides being fun and thus motivational.}
\end{singlespace}

\textit{\textbf{Keywords}: Physics Teaching, High School, Particle Physics, Learning, Digital Games.}
