� Deus, criador de tudo.

� minha Fam�lia. Minha esposa Raquel por toda paci�ncia e colabora��o, pois sem voc� eu n�o conseguiria ter for�as suficientes para chegar at� aqui.

Aos meus pais Jos� e Elena, meus espelhos de sabedoria, agrade�o por todos os ensinamentos e pela educa��o. � minha irm� Ana Caroline, por todo o apoio que me destes. 

� minha orientadora Vanessa Carvalho de Andrade, por toda colabora��o e parceria desde a �poca do PET, sou grato pela disponibilidade e pelos conselhos.

Aos meus colegas de mestrado: Mateus Gross, Ricardo Nonato, Let�cia, Andr� Chaul, Jos� Alex e Adriana, por todas parcerias e colabora��es.

Aos meus grandes professores e mestres: Jaqueline Mendes, Am�rico Silvano, Ant�nio Carlos Pedroza, Oyanarte Portilho, Paulo Eduardo Narciso, Arsen Melikyan, Vanessa Carvalho de Andrade, Ronni Amorim e Wytler Cordeiro dos Santos.

� equipe diretiva do Centro Educacional 15 de Ceil�ndia, por ter apoiado o projeto e incentivado a aplica��o do meu produto.

Ao SESI, por ter dado � oportunidade de realizar um curso de capacita��o em Salvador no qual foi de suma import�ncia para a escolha do tema deste trabalho.

A Universidade de Bras�lia, por ter dado a oportunidade de realizar tanto a gradua��o quanto a p�s-gradua��o.

� Sociedade Brasileira de F�sica pelo programa de Mestrado Nacional Profissional em Ensino de F�sica.

� CAPES pelo apoio financeiro.

Enfim, aos meus alunos, pois aprendo muito mais do que ensino...

